\chapter{Conclusão}
\label{c.conclusao}
Com o avanço da computação, é interessante introduzir tecnologias novas na meteorologia, pois apesar de já existirem algumas soluções precisas para a simulação pluviométrica e previsão do tempo, analisar outras opções que possam utilizar um hardware mais atual para o processamento de dados é sempre válido. Desenvolver uma abordagem alternativa e inovadora para a solução do problema de prever a chuva é bastante desafiador, pois um bom grau de precisão é necessário para que se possa ter algum uso na agronomia ou outras áreas afins.

Um dos principais desafios encontrados foi a falta de dados históricos. No futuro, como as estações pluviométricas automáticas vêm gravando mais dados continuamente, o problema tende a ser muito menor. Programar o script para processar e manipular milhares de dados foi trabalhoso, mas o MATLAB agilizou todo o trabalho.

Para projetos futuros, ainda tem espaço para novas implementações, como por exemplo a magnitude da chuva e melhorias de precisão ao aumentar a base histórica de dados. No entanto, simular se um dia vai ser chuvoso ou seco já é bastante desafiador, então determinar a magnitude da chuva para cada dia chuvoso o torna ainda mais complexo.

Como aspectos positivos, o trabalho teve êxito ao realizar a simulação pluviométrica computacional com uma taxa de precisão média de valor bastante satisfatório. Cerca de 67\% dos dias foram previstos corretamente, com destaque ao mês de janeiro onde a maioria dos dias chuvosos e secos foram simulados com acerto. No ano todo, 98 dias foram chuvosos, enquanto para a simulação foram 101 dias com chuva, números muito próximos. A linguagem de programação MATLAB, utilizada para programar o script da simulação, teve um desempenho muito satisfatório pois foi capaz de processar dezenas de milhares de dados quase que instantaneamente e em um hardware de baixo custo.