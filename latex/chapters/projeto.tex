\pagestyle{simple}
\chapter{Projeto do Software}
\label{c.projeto}

O projeto do software é uma solução para a simulação das condições de chuva durante um determinado período para uma cidade qualquer que tenha disponível dados históricos pluviométricos, de preferência maiores do que 20 anos.

O modelo usado na simulação utiliza a Cadeia de Markov para gerar as probabilidades de ocorrência de chuva, para então, em seguida, determinar se o dia vai ser chuvoso ou seco. Para isso, deve ser analisado previamente um certo intervalo de dias anteriores ao dia a ser previsto. Como dito anteriormente, o trabalho adotou que se a precipitação pluviométrica de certo dia for maior ou igual a 0, então ele é considerado chuvoso. Caso contrário, é considerado seco.

A implementação em MATLAB é possível devido a facilidade de trabalho com vetores e matrizes que a ferramenta proporciona. Dando como \emph{input} as variáveis da tabela, cada uma organizada dentro de um vetor, foi viável a elaboração de um script para simular os dias chuvosos e secos de um ano inteiro.

A cidade analisada foi Bauru, do estado de São Paulo, que tem disponível dados históricos desde 2002 através da estação automática A705. Para que seja possível realizar uma comparação de resultados no final, a fim de verificar a precisão da simulação realizada neste trabalho, o ano escolhido para ser simulado foi 2021.
\section{Especificação}
\label{s.especificacao}

Para realizar a simulação, com base no modelo analisado, é necessário efetuar um processo de geração das séries de dias secos e chuvosos através da comparação das probabilidades condicionais \emph{P(C|S)} e \emph{P(C|C)} com um número aleatório \emph{x} entre 0 e 1. O processo é inicializado em qualquer dia do mês simulado, desta forma, a definição do estado inicial (seco ou chuvoso) do dia anterior é obtida a partir de uma fonte meteorológica confiável \cite{artigo_modelo}. Para os demais dias, é feito da seguinte maneira:

\begin{enumerate}
  \item verifica se o dia anterior foi chuvoso ou seco;
  \item se o dia anterior for seco, gera-se um número aleatório \emph{x} entre 0 e 1, e o compara com a probabilidade condicional P(C|S) gerada para o dia atual analisado. Já se o dia anterior for chuvoso, compara-se o número aleatório \emph{x} com a probabilidade condicional P(C|C) obtida para o dia atual analisado. Essa comparação é feita da seguinte forma:
  \begin{enumerate}[label=(\alph*)]
   \item se x $\leq$ P(C|S) ou P(C|C), o estado do dia atual é considerado chuvoso; 
   \item se x > P(C|S) ou P(C|C), o estado do dia atual é considerado seco.
  \end{enumerate}
\end{enumerate}

As probabilidades condicionais são definidas com base na Cadeia de Markov, com suas fórmulas descritas abaixo:

\[
    P(C|S)=\frac{N(C|S)}{N(C|S)+N(C|S)}=\frac{N(C|S)}{N(S)}
\]
\[
    P(C|C)=\frac{N(C|C)}{N(S|C)+N(C|C)}=\frac{N(C|C)}{N(C)}
\]

\section{Tecnologias Pesquisadas e Escolhidas}
\label{s.tecnologias}

Existem diversas tecnologias que são capazes de realizar a simulação pluviométrica de maneira satisfatória. De todas as analisadas, a ferramenta escolhida foi o MATLAB, uma linguagem de programação de alta performance voltada para o cálculo numérico computacional.

\subsection{MATLAB}
\label{ss.matlab}

Para um projeto que envolve muitos cálculos numéricos e manipulações de extensas matrizes e vetores, o software MATLAB é líder no segmento \cite{matlab}. Trata-se de um software robusto que consegue resolver muitos problemas em um tempo muito menor de programação quando comparado a outras linguagens de programação.
A versão escolhida para realizar este trabalho foi a R2021a, lançada em março de 2021, porém existem muitas outras versões desde o ano de 1984. Trata-se de um software proprietário pago, mas existem licenças exclusivas para estudantes e professores.

\begin{figure}[H]
    \caption{\small Demonstração de sintaxe na IDE do MATLAB.}
	\centering
	\includegraphics[width=\textwidth]{figs/matlab-syntax.png}
	\label{f.matlab-syntax}
	\legend{\small Fonte: \cite{matlab-jetbrains}.}
\end{figure}

\subsection{Mathematica}
\label{ss.mathematica}

O Mathematica é outra ferramenta que permite resolver problemas matemáticos com certa facilidade, utilizando de várias bibliotecas de programação para auxiliar na resolução de problemas de ciências exatas.
Ele é utilizado em muitas áreas da computação, como redes neurais, aprendizado de máquina, processamento de imagens, geometria, ciência de dados, visualizações e diversas outras \cite{mathematica}.

\begin{figure}[H]
    \caption{\small Demonstração de sintaxe na IDE do Mathematica.}
	\centering
	\includegraphics[width=\textwidth]{figs/mathematica-ide.png}
	\label{f.matlab-syntax}
	\legend{\small Fonte: \cite{mathematica}.}
\end{figure}

\subsection{GNU Octave}
\label{ss.gnu}

Em termos de compatibilidade e capacidade computacional, o GNU Octave é considerado a melhor alternativa ao MATLAB \cite{matlab-alts}. A maioria dos projetos desenvolvidos para o MATLAB também conseguem ser executados no Octave. Além disso, ele roda em qualquer sistema operacional sem nenhuma modificação. 

Ele pode lidar com grandes dados matemáticos e tem ferramentas de plotagem e visualização. Além disso, é um software de código aberto e foi desenvolvido principalmente para cálculos numéricos lineares e não lineares complexos. Ele pode executar trabalhos interativos em lote e tem compatibilidade com scripts MATLAB além de outras linguagens como Java, C++ ou Fortran.

\begin{figure}[H]
	\caption{\small Demonstração de sintaxe na IDE do GNU Octave.}
	\centering
	\includegraphics[width=\textwidth]{figs/octave-ide.png}
	\label{f.matlab-syntax}
	\legend{\small Fonte: \cite{octave-ide}.}
\end{figure}